\documentclass[a4paper]{article}

\usepackage[spanish]{babel}
\usepackage[utf8]{inputenc}
\usepackage{graphicx}
\usepackage{amsmath}
\usepackage[T1]{fontenc}
\usepackage{hyperref}


\title{Proyecto de investigación sobre detección de caras mediante OpenCV} 
\author{Alejandro Rodríguez Rodríguez \and Pablo Cano Navajas \and Salvador Caballero Macías}

\begin{document}

\maketitle

\newpage
\tableofcontents

\newpage
\section*{Resumen}

Este documento detalla las diferentes implementaciones que se han llevado a cabo durante la implementación de las mismas, asi como los resultados obtenidos y las conclusiones a las que se ha llegado. El proyecto se ha realizado empleando el repositorio\cite{1} que se detalla en el libro escogido\cite{2}.
De la documentacion escogida, se ha seleccionado el capítulo número 5, que trata sobre la detección de caras, tanto en imágenes como en captura en tiempo real, así como las implementaciones de mejoras de los algoritmos que se tratan en el libro para detectar un mayor número de caras o distintos objetos que cumplan unas restricciones que se impongan.

\section{Introducción}

El capítulo 5 del libro \textit{Learning OpenCV 4 Computer Vision with Python 3 Third Edition} \cite{1} se centra en la \textbf{detección y reconocimiento de caras} [1]. Este capítulo introduce la funcionalidad de OpenCV para estas tareas, junto con los archivos de datos que definen tipos particulares de objetos rastreables [2]. Se exploran los \textbf{clasificadores de cascada Haar}, que analizan el contraste entre regiones de imagen adyacentes para determinar si una imagen o subimagen coincide con un tipo conocido [2].

La detección de rostros es un problema fundamental en el campo de la visión por computador, con aplicaciones que abarcan desde sistemas de seguridad
hasta interfaces hombre-máquina1. Un método clásico y ampliamente utilizado
para la detección de rostros es el uso de clasificadores en cascada de Haar,
implementados eficientemente en bibliotecas como OpenCV2. Estos clasificadores
analizan el contraste entre regiones adyacentes de una imagen mediante un
conjunto de características similares a Haar organizadas en una estructura en
cascada para una rápida evaluación de posibles regiones de rostro3. Si bien esta
técnica ha demostrado ser efectiva en muchas situaciones, su rendimiento puede
verse afectado por diversos factores, como la escala, la orientación y las condiciones
de detección4. En este trabajo, exploramos una mejora en la detección
de rostros utilizando los clasificadores en cascada de Haar mediante la optimizaci
ón de los parámetros del proceso de detección, buscando un equilibrio entre
la sensibilidad del detector y la reducción de falsos positivos.

\section{Planteamiento teórico}

En esta sección se detalla el planteamiento teórico del capítulo 5 del libro\cite{1}, que se centrará en los fundamentos de las cascadas de Haar mediante el uso de OpenCV para la detección de caras. Se dividirá en 3 subapartados, correspondientes a los subapartados del propio capítulo:

Los clasificadores en cascada de Haar funcionan mediante la aplicación de
una serie de clasificadores, cada uno entrenado para descartar la mayoría de
las regiones negativas mientras retiene las regiones que contienen el objeto de
intés (en este caso, rostros)5. Estos clasificadores se basan en la evaluación
de características simples, computacionalmente eficientes, que codifican diferencias
en la intensidad de píxeles entre regiones rectangulares adyacentes6. Para
detectar rostros en imágenes de diferentes tamaños, se emplea una ventana de
detección que se desliza sobre la imagen a diferentes escalas, generando múltiples
subventanas para su análisis7.
En OpenCV, la detección de múltiples escalas de rostros se implementa
mediante la función detectMultiScale del objeto cv2.CascadeClassifier8.
Esta función toma como entrada una imagen y varios parámetros que controlan
el proceso de búsqueda de objetos. Dos de los parámetros más influyentes en el
rendimiento del detector son scaleFactor y minNeighbors.

\subsection{Conceptualizing Haar Cascade Data}

El concepto de clasificación de objetos y el seguimiento de su ubicación buscan identificar qué constituye una parte reconocible de un objeto . Las imágenes fotográficas pueden contener muchos detalles, pero estos detalles pueden ser inestables debido a variaciones en la iluminación, el ángulo de visión, la distancia de visión, el movimiento de la cámara y el ruido digital . Afortunadamente, para la clasificación, no todas las diferencias en los detalles físicos son relevantes .

Las \textbf{características tipo Haar} son un tipo de característica que se aplica a menudo a la detección de rostros en tiempo real . Estas características describen el patrón de contraste entre regiones de imagen adyacentes . Por ejemplo, los bordes, los vértices y las líneas delgadas generan un tipo de característica . Algunas características son distintivas en el sentido de que típicamente ocurren en una cierta clase de objeto (como una cara) pero no en otros objetos . Estas características distintivas se pueden organizar en una jerarquía, llamada \textbf{cascada}, en la que las capas superiores contienen características de mayor distinción, lo que permite que un clasificador rechace rápidamente los sujetos que carecen de estas características .

Las características pueden variar según la escala de la imagen y el tamaño del vecindario dentro del cual se evalúa el contraste, llamado \textbf{tamaño de ventana} . Para hacer que un clasificador de cascada Haar sea \textbf{invariante a la escala}, el tamaño de la ventana se mantiene constante pero las imágenes se reescalan varias veces; de esta manera, a algún nivel de reescalado, el tamaño de un objeto (como una cara) puede coincidir con el tamaño de la ventana . La imagen original y las imágenes reescaladas juntas se denominan \textbf{pirámide de imágenes}, y cada nivel sucesivo en esta pirámide es una imagen reescalada más pequeña . OpenCV proporciona un clasificador invariante a la escala que puede cargar una cascada Haar desde un archivo XML en un formato particular . Internamente, este clasificador convierte cualquier imagen dada en una pirámide de imágenes .

\subsection{Using OpenCV to Perform Face Detection}

El código fuente de OpenCV 4, o una instalación preempaquetada, debería contener una subcarpeta llamada \texttt{data/haarcascades} . Esta carpeta contiene archivos XML que pueden ser cargados por una clase de OpenCV llamada \texttt{cv2.CascadeClassifier} . Una instancia de esta clase interpreta un archivo XML dado como una cascada Haar, que proporciona un modelo de detección para un tipo de objeto como una cara . \texttt{cv2.CascadeClassifier} puede detectar este tipo de objeto en cualquier imagen, ya sea una imagen fija de un archivo o una serie de fotogramas de un archivo de video o una cámara de video .

Para realizar la detección de caras, se puede crear un script básico que cargue un clasificador de cascada Haar para la detección de rostros y luego aplique este clasificador a una imagen . El método clave para realizar la detección de caras es \texttt{detectMultiScale}, que se aplica a una imagen en escala de grises . Los parámetros de \texttt{detectMultiScale} incluyen \texttt{scaleFactor} y \texttt{minNeighbors} . El argumento \texttt{scaleFactor}, que debe ser mayor que 1.0, determina la relación de reducción de escala de la imagen en cada iteración del proceso de detección de rostros . El argumento \texttt{minNeighbors} es el número mínimo de detecciones superpuestas que se requieren para conservar un resultado de detección .

También es posible realizar la detección de caras en un video utilizando una cascada Haar para rostros y otra para ojos. El proceso implica capturar fotogramas de una cámara, convertirlos a escala de grises y luego aplicar el detector de rostros. Para cada rostro detectado, se puede definir una región de interés (ROI) y aplicar un detector de ojos dentro de esa ROI .

\subsection{Improving the Haar Cascade Classifier}

La efectividad de un clasificador de cascada Haar puede verse afectada por los parámetros utilizados en la función \texttt{detectMultiScale}, como \texttt{scaleFactor} y \texttt{minNeighbors} . El \texttt{scaleFactor} influye en la robustez a diferentes tamaños de rostro, mientras que \texttt{minNeighbors} ayuda a reducir los falsos positivos al requerir múltiples detecciones superpuestas . Ajustar estos parámetros mediante experimentación puede mejorar el rendimiento del detector en diferentes condiciones de iluminación y para diferentes sujetos . Además, el uso de múltiples clasificadores en cascada, como uno para la detección frontal de rostros y otro para la detección de ojos dentro de la región facial detectada, puede aumentar la precisión de la detección de características específicas .

\section{Implementación}

\subsection{Conceptualizing Haar Cascade Data}

\textbf{implementacion1}

Para utilizar un clasificador de cascada Haar en OpenCV para la detección de caras, el primer paso es cargar el archivo XML de la cascada utilizando la función \texttt{cv2.CascadeClassifier()} :

\begin{verbatim}
import cv2

face_cascade = cv2.CascadeClassifier('./cascades/haarcascade_frontalface_default.xml')
\end{verbatim}

\subsection{Using OpenCV to Perform Face Detection}

En esta sección se explica cómo detectar caras tanto en imagenes como en una entrada de vídeo en tiempo real (e.g. una cámara de vídeo).
OpenCV proporciona herramientas avanzadas para el procesamiento de imágenes y la detección de objetos mediante el uso de clasificadores en cascada de Haar.\newline
La detección de caras se realiza utilizando archivos XML que contienen los datos preentrenados para identificar características faciales específicas. Estos clasificadores se cargan mediante la función \texttt{cv2.CascadeClassifier}, que permite aplicar los modelos a imágenes o fotogramas de vídeo.
Para detectar caras en imágenes estáticas, se utiliza el script \texttt{\detokenize{0_stillImageFaceDetection.ipynb}}. Este script carga una imagen, aplica un clasificador en cascada y detecta las caras presentes en la misma. El proceso incluye los siguientes pasos:

\begin{itemize}
    \item Carga de la imagen mediante \texttt{cv2.imread}.
    \item Conversión de la imagen a escala de grises para optimizar el rendimiento del clasificador.
    \item Uso del clasificador \texttt{haarcascade\_frontalface\_default.xml} para detectar caras.
\end{itemize}

El script \texttt{1\_cameraFaceDetection.ipynb} implementa la detección de caras en tiempo real utilizando la cámara del dispositivo. Este proceso incluye:
\begin{itemize}
    \item Captura de vídeo en tiempo real mediante \texttt{cv2.VideoCapture}.
    \item Aplicación del clasificador en cascada a cada fotograma del vídeo.
    \item Detección de características adicionales, como ojos, utilizando el archivo \texttt{haarcascade\_eye.xml}.
\end{itemize}

\subsection{Improving the Haar Cascade Classifier}

implementacion3

\section{Experimentación}

\section{Manual de usuario}

Para el correcto funcionamiento de los scripts, es necessario la instalación de los siguientes paquetes:

\begin{itemize}
    \item \textbf{Windows 7, MacOS 10.7 o superior}.
    \item \textbf{Python 3.8 o superior}. Para instalar Python, se recomienda instalar la versión más reciente accediendo a la página web \href{https://www.python.org/downloads/}{Python.org}. Haremos click en el botón de descarga y se descargará automáticamente el ejecutable. Si fuese necesario otra versión de python, en la misma págna se puede descargar la versión que se necesite.\\
    Si se tiene instalado un sistema operativo distinto a windows, se puede acceder al enlace de \href{https://www.python.org/downloads/macos/}{versiones macOS} y descargar la que sea necesaria siempre que cumppla con los requisitos de instalación.
    \item \textbf{OpenCV 4.0 o superior}. Para instalar la version 4.0 o superior de OpenCV, se recomienda usar el gestor de paquetes \texttt{pip}. Para ello, se abre una terminal y se ejecuta el comando \texttt{pip install opencv-python}. Si la máquina opera con macOS, se recomienda usar \texttt{brew}, escribiendo el comando \texttt{brew install opencv}.
    \item \textbf{NumPy 1.16 o superior}. Para instalarlo, en una ventana de comandos escribiremos \texttt{pip install numpy} (para un entorno Windows) o \texttt{brew install numpy} (para un entorno macOS).
    \item \textbf{Scipy 1.1 o superior}. Para instalarlo, en una ventana de comandos escribiremos \texttt{pip install scipy} (para un entorno Windows) o \texttt{brew install scipy} (para un entorno macOS).
\end{itemize}

\section{Conclusiones}

\section{Autoevaluación de cada miembro}

\subsection{Autoevaluación de Alejandro}

\subsection{Autoevaluación de Pablo}

\subsection{Autoevaluación de Salvador}

\section{Tabla de tiempos}

\subsection{Tabla de tiempos del grupo}

\subsection{Tabla de tiempos de Alejandro}

\subsection{Tabla de tiempos de Pablo}

\subsection{Tabla de tiempos de Salvador}

\begin{thebibliography}{99}
\bibitem{1} \textit{Repositorio del proyecto}, disponible en \href{https://github.com/PacktPublishing/Learning-OpenCV-4-Computer-Vision-with-Python-Third-Edition}{GitHub}
\bibitem{2} Joseph Howse, Joe Minichino, \textit{Learning OpenCV 4 Computer Vision with Python 3}", Third edition, Packt Publishing, pp. 1-372, 2020
\end{thebibliography}

\end{document}