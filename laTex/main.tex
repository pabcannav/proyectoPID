\documentclass[a4paper]{article}

\usepackage[spanish]{babel}
\usepackage[utf8x]{inputenc}
\usepackage{graphicx}
\usepackage{amsmath}
\usepackage[T1]{fontenc}
\usepackage{hyperref}


\title{Título del documento aquí} 
\author{Alejandro Rodríguez Rodríguez, Pablo Cano Navajas, Salvador Caballero Macías}

\begin{document}

\maketitle

\newpage
\tableofcontents

\newpage
\section{Resumen}

Este documento detalla las diferentes implementaciones que se han llevado a cabo durante la implementación de las mismas, asi como los resultados obtenidos y las conclusiones a las que se ha llegado. El proyecto se ha realizado empleando el repositorio\textsuperscript{1} que se detalla en el libro escogido\textsuperscript{2}.
De la documentacion escogida, se ha seleccionado el capítulo número 5, que trata sobre la detección de caras, tanto en imágenes como en captura en tiempo real, así como las implementaciones de mejoras de los algoritmos que se tratan en el libro para detectar un mayor número de caras o distintos objetos que cumplan unas restricciones que se impongan.

\section{Introducción}

\section{Planteamiento teórico}

En esta sección se detalla el planteamiento teórico del capítulo 5 del libro\textsuperscript{2}. Se dividirá en 3 subapartados, correspondientes a los subapartados del propio capítulo:

\subsection{\textit{Conceptualizaing Haar Cascade Data}}

subseccion1

\subsection{\textit{Using OpenCV to Perform Face Detection}}

En esta sección se explica cómo detectar caras tanto en imagenes como en una entrada de vídeo en tiempo real (e.g. una cámara de vídeo). A su vez, se divide en 3 subapartados:

\subsubsection{\textit{Performing Face Detection on a Still Image}}

\subsubsection{\textit{Performing Face Detection on a Video}}

\subsubsection{\textit{Performing Face Recognition}}

\subsection{\textit{Improving the Haar Cascade Classifier}}

subseccion3

\section{Implementación}

\section{Experimentación}

\section{Manual de usuario}

\section{Conclusiones}

\section{Autoevaluación de cada miembro}

\subsection{Autoevaluación de Alejandro}

\subsection{Autoevaluación de Pablo}

\subsection{Autoevaluación de Salvador}

\section{Tabla de tiempos}

\subsection{Tabla de tiempos del grupo}

\subsection{Tabla de tiempos de Alejandro}

\subsection{Tabla de tiempos de Pablo}

\subsection{Tabla de tiempos de Salvador}

\begin{thebibliography}{99}
\bibitem{1} \textit{Repositorio del proyecto}, disponible en \href{https://github.com/PacktPublishing/Learning-OpenCV-4-Computer-Vision-with-Python-Third-Edition}{GitHub}
\bibitem{2} Joseph Howse, Joe Minichino, \textit{Learning OpenCV 4 Computer Vision with Python 3}", Third edition, Packt Publishing, pp. 1-372, 2020
\end{thebibliography}

\end{document}